% !TeX root = ../install-latex-guide-zh-cn.tex

\chapter{版本信息}

\section*{v2020.11.1}

\begin{enumerate}
  \item Overleaf 后台更新为 \TeX Live 2020
  \item 补充卸载方法
  \item 补充 PPA 反向代理安装 \TeX Studio 的方法
\end{enumerate}

\section*{v2020.10.1}

\begin{enumerate}
  \item 增加北京外国语大学镜像链接
  \item 修正 Windows 10 中用 \texttt{sudo} 命令的错误
  \item 完善 Windows 下手动删除涉及的注册表位置
\end{enumerate}

\section*{v2020.8.2}

\begin{enumerate}
  \item 增加验证 sha512 的内容
  \item 将本文提交至 CTAN 而进行必要更改
  \item 利用 \texttt{texdoc} 调出本手册
  \item 给出直接下载 Mac\TeX 的链接
\end{enumerate}

\section*{v2020.7.1}

\begin{enumerate}
  \item \TeX works 拼写检查调整
  \item VS Code 是开源软件
  \item 改摘要为前言并增加内容
  \item 增加调出宏包手册内容
\end{enumerate}

\section*{v2020.6.1}

\begin{enumerate}
  \item 重述大陆地区的源的相关内容
  \item 为适应 Ubuntu 20.04 调整某些内容
  \item 调整有关管理员权限的内容
  \item 调整有关编辑器的内容
  \item 借助网站加速下载 TeXstudio
  \item 提示卸载时的权限
  \item 给出 make.bat 与 makefile
  \item 增加 MD5 的说明
  \item 更正设置主文档的错误
\end{enumerate}

\section*{v2020.5.1}

\begin{enumerate}
  \item 更新 MiK\TeX{} 相关网址
  \item 更新 至 \TeX{} Live 2020
  \item 更新 Windows 10 手动卸载 \TeX{} Live 的内容
  \item 调整 \TeX studio 图标大小
  \item 更新镜像网址
  \item 更改 openoffice 网址
  \item 更新 Windows 10 升级宏包描述
  \item 更新 Overleaf 的介绍
\end{enumerate}

\section*{v2020.4.1}

\begin{enumerate}
  \item 简要介绍 WSL 辅助程序
  \item 添加 Overleaf 学习与帮助
  \item 添加两处与 WSL 有关的内容
  \item 增加几处软件菜单栏的讲解
  \item 更清晰查看 Windows 10 环境变量的代码
  \item Windows 10 直接以管理员身份打开命令行安装
  \item 调整标点
\end{enumerate}

\section*{v2020.3.1}

\begin{enumerate}
    \item 摘要中明确提到 C\TeX{} 套装和发行版、编辑器
    \item 更改 Windows 10 安装 \TeX{} Live 的描述, 如路径名, 文件扩展名等
    \item 将文件名要求写入注释
    \item 添加 \TeX studio 数学符号表和插入、改写调整
    \item 解决 \LaTeXe 版本不匹配导致 \texttt{xelatex} 失败的问题
    \item 添加打开 \textsf{cmd} 的方法
    \item 改写 \TeX studio 显示行号的设置
    \item 更新 MiK\TeX{} 相关网址
    \item 添加调整 \TeX works 字体的内容
    \item 添加 Windows 中手动卸载 \TeX{} Live 的方法
    \item 添加 Overleaf 中文网址
\end{enumerate}

\section*{v2020.2.1}

\begin{enumerate}
    \item 添加版本信息列表
    \item 改变 Windows 10 的 \TeX studio 的下载网址
    \item 添加 macOS 中 \texttt{xelatex} 用字体名调用字体的方法
    \item 详述 WSL 中不懂 vim 的处理方法和阅读宏包手册的方法
    \item 强调路径为不带空格的英文
    \item 订正若干拼写错误
    \item 给出 \texttt{jdk} 环境变量的处理
    \item 添加北京交通大学镜像
    \item 注明安装需卸载国产安全软件
    \item 更正 \TeX works 自动补全的教程网址
    \item 给出更换 2345 好压的建议
    \item 补充 Overleaf 升级内容
\end{enumerate}